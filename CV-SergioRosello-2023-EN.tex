%%%%%%%%%%%%%%%%%%%%%%%%%%%%%%%%%%%%%%%%%
% Developer CV
% LaTeX Template
% Version 1.0 (28/1/19)
%
% This template originates from:
% http://www.LaTeXTemplates.com
%
% Authors:
% Jan Vorisek (jan@vorisek.me)
% Based on a template by Jan Küster (info@jankuester.com)
% Modified for LaTeX Templates by Vel (vel@LaTeXTemplates.com)
%
% License:
% The MIT License (see included LICENSE file)
%
%%%%%%%%%%%%%%%%%%%%%%%%%%%%%%%%%%%%%%%%%

%----------------------------------------------------------------------------------------
%	PACKAGES AND OTHER DOCUMENT CONFIGURATIONS
%----------------------------------------------------------------------------------------

\documentclass[9pt]{developercv} % Default font size, values from 8-12pt are recommended

%----------------------------------------------------------------------------------------

\begin{document}

%----------------------------------------------------------------------------------------
%	TITLE AND CONTACT INFORMATION
%----------------------------------------------------------------------------------------

\begin{minipage}[t]{0.45\textwidth} % 45% of the page width for name
	\vspace{-\baselineskip} % Required for vertically aligning minipages
	
	% If your name is very short, use just one of the lines below
	% If your name is very long, reduce the font size or make the minipage wider and reduce the others proportionately
	\colorbox{black}{{\HUGE\textcolor{white}{\textbf{\MakeUppercase{Sergio}}}}} % First name
	
	\colorbox{black}{{\HUGE\textcolor{white}{\textbf{\MakeUppercase{Roselló}}}}} % Last name
	
	\vspace{6pt}
	
	{\huge DevSecOps} % Career or current job title
\end{minipage}
\begin{minipage}[t]{0.275\textwidth} % 27.5% of the page width for the first row of icons
	\vspace{-\baselineskip} % Required for vertically aligning minipages
	
	% The first parameter is the FontAwesome icon name, the second is the box size and the third is the text
	% Other icons can be found by referring to fontawesome.pdf (supplied with the template) and using the word after \fa in the command for the icon you want
	\icon{MapMarker}{12}{Madrid}\\
	\icon{Phone}{12}{+34 685106958}\\
	\icon{At}{12}{\href{mailto:sergio-rosello@hotmail.com}{sergio-rosello@h..l.com}}\\	
\end{minipage}
\begin{minipage}[t]{0.275\textwidth} % 27.5% of the page width for the second row of icons
	\vspace{-\baselineskip} % Required for vertically aligning minipages
	
	% The first parameter is the FontAwesome icon name, the second is the box size and the third is the text
	% Other icons can be found by referring to fontawesome.pdf (supplied with the template) and using the word after \fa in the command for the icon you want
	\icon{Globe}{12}{\href{http://www.sergiorosello.com}{sergiorosello.com}}\\
	\icon{Github}{12}{\href{https://github.com/sergiorosello}{github.com/sergiorosello}}\\
	\icon{Instagram}{12}{\href{https://www.instagram.com/sergiorosello/}{sergiorosello}}\\
\end{minipage}

\vspace{0.5cm}

%----------------------------------------------------------------------------------------
%	INTRODUCTION, SKILLS AND TECHNOLOGIES
%----------------------------------------------------------------------------------------

\cvsect{Who Am I?}

\begin{minipage}[t]{0.4\textwidth} % 40% of the page width for the introduction text
	\vspace{-\baselineskip} % Required for vertically aligning minipages
	I'm fueled by curiosity, it is the reason I wake up every day.
	This means I will never be able to stop learning or observing.
	Attitudes like teamwork, perseverance and passion define me as a person.
	Aptitudes are developed from that foundation onward.\\
\end{minipage}
\hfill % Whitespace between
\begin{minipage}[t]{0.5\textwidth} % 50% of the page for the skills bar chart
	\vspace{-\baselineskip} % Required for vertically aligning minipages
	\begin{barchart}{5.5}
		\baritem{Golang}{80}
		\baritem{Kubernetes}{60}
		\baritem{Terraform}{60}
		\baritem{Git}{80}
		\baritem{GitLab CI/CD}{80}
		\baritem{AWS}{50}
		%\baritem{GNU/Linux}{80}
		%\baritem{Vim/Neovim}{90}
	\end{barchart}
\end{minipage}

%----------------------------------------------------------------------------------------
%	EXPERIENCE
%----------------------------------------------------------------------------------------

\cvsect{Experience}

\begin{entrylist}
	\entry
		{2020 -- Current\\\footnotesize{Full time}}
		{DevSecOps}
		{Scalefast}
		{I am focused in ensuring that the security of our applications meets Scalefasts standards throught the whole SDLC.
		These are some of the projects I have been involved in to accomplish my mission:
		\textbf{Kubernetes} admission controllers to prevent vulnerable images from being deployed into our infrastructure.
		Dynamic Pluggable \textbf{Gitlab CI Security pipelines} to perform vulnerability, secrets and missconfiguration analysis.
		Secret lifecycle management with \textbf{Hashicorp Vault}.
		Container image securization and analysis with \textbf{ECR} and \textbf{Trivy}.
		Integrated Shift-Left security tool that includes \textbf{DAST, CSA, Secrets} checks on source code.
		\\ \texttt{Kubernetes}\slashsep\texttt{Bash}\slashsep\texttt{Golang}\slashsep\texttt{OPA Gatekeeper}\slashsep\texttt{Hashicorp Vault}\slashsep\texttt{Terraform}\slashsep\texttt{AWS}\slashsep\texttt{Gitlab CI/CD}}
	\entry
		{2018 -- 2020\\\footnotesize{Full time}}
		{QA Engineer}
		{Wave Location Technologies}
		{My role is to ensure the quality of every app under the company's name.
		To accomplish this, I have implemented a end-to-end test automation framework, and a CI server with Jenkins.\\ \texttt{Ruby}\slashsep\texttt{Bash}\slashsep\texttt{Jenkins}\slashsep\texttt{Linux}\slashsep\texttt{Gherkin}\slashsep\texttt{Appium}}
\end{entrylist}


%----------------------------------------------------------------------------------------
%	Publications and conferences
%----------------------------------------------------------------------------------------
\cvsect{Publications and Conferences}
\begin{entrylist}
	\entry
	  {15-06-2023}
	  {\href{https://play.vidyard.com/uvjJ6wyme9XUcCjY6U2pTN}{Building robust applications}}
    	  {AWS Summit Madrid}
    	  {Co-host of this conference talk. I focused on security best practices throughout the SDLC}
	\entry
	  {12-01-2023}
	  {\href{https://scalefast.engineering/owasp-zap-api-dast-analysis-with-talos-523c63b860a0}{OWASP ZAP API DAST analysis with TALOS}}
          {Scalefast engineering tech blog}
          {An in-depth review of how TALOS DAST module works, and how we use it}
	\entry
	  {12-01-2023}
	  {\href{https://scalefast.engineering/introducing-talos-fd12c450b48}{Introducing TALOS}}
          {Scalefast engineering tech blog}
          {A first introduction of TALOS, a Shift-Left security tool aimed at providing a suite of security scanners to developers}
\end{entrylist}


%----------------------------------------------------------------------------------------
%	EDUCATION
%----------------------------------------------------------------------------------------
\cvsect{Education}
\begin{entrylist}
	\entry
	  {present}
	  {Let's Go Further Golang development course}
    {Let's go further - Alex Edwards}
    {Develop production-grade golang with a focus on web servers.}
	\entry
	  {2019 -- 2021}
	  {Master's Degree in Cybersecurity}
    {UNED}
    {Studied topics such as forensic analysis and ethical hacking, malware analysis, and machine learning applied to Cybersecurity.}
	\entry
		{2014 -- 2018}
		{Bachelor's Degree in Software Engineering}
		{U-TAD}
		{The Degree is very practically focused. This means I have had a hands on approach to everything I have learned here.
		My Bachelor's Degree Final Project was focused on an analysis of double authentication methods on GNU/Linux systems with a special focus on PAM}
\end{entrylist}


%----------------------------------------------------------------------------------------
%	MINDSET
%----------------------------------------------------------------------------------------
\cvsect{Personality}

\begin{entrylist}
	\entry
		{Currently}
		{ISFP}
		{Myers and Briggs'}
		{I focus on the job at hand, always prioritizing personal relationships first.
		Stoic.
		I tend to pause for a better answer.
		Self-improvement is always on my To-Do list.
		Less is more.}
\end{entrylist}

%----------------------------------------------------------------------------------------
%	ADDITIONAL INFORMATION
%----------------------------------------------------------------------------------------

\begin{minipage}[t]{0.3\textwidth}
	\vspace{-\baselineskip} % Required for vertically aligning minipages

	\cvsect{Languages}
	
	\textbf{Spanish} - native\\
	\textbf{Catalan} - native\\
	\textbf{English} - C1 - Cambridge
\end{minipage}
\hfill
\begin{minipage}[t]{0.3\textwidth}
	\vspace{-\baselineskip} % Required for vertically aligning minipages
	
	\cvsect{Hobbies}

	I enjoy learning about GNU/Linux and configuring my system (Arch + dwm).\\
	Photography. I will go everywhere with my camera.\\
	Spending time with my family.\\
	Sport is a key part of my life.\\
\end{minipage}
\hfill
\begin{minipage}[t]{0.3\textwidth}
	\vspace{-\baselineskip} % Required for vertically aligning minipages
	
	\cvsect{OPEN SOURCE}

	I believe Software should be Open Source as it serves as a learning platform for aspiring developers as much as a community effort to break current boundaries and limitations.
\end{minipage}

%----------------------------------------------------------------------------------------
%	OTHER EVENTS - CHRONOLOGICAL
%----------------------------------------------------------------------------------------

\cvsect{Other events - Chronological order}

\begin{entrylist}
	\entry
	  {2019}
    	  {Photography exhibition}
    	  {PRINTO - C/ Dr. Vicente Pallarés, 24 46021, Valencia}
    	  {A collection of my street photographs.}
  	\entry
          {25-01-2019}
	  {Setup an automated testing server with Jenkins and AWS Device farm}
          {Faun - Medium}
          {Step by step configuration of a Jenkins, AWS - Device Farm CI server}
	\entry
          {01-12-2018}
	  {Leverage the power of open source and don’t reinvent the wheel}
	  {Hacker Noon - Medium}
          {How not to waste time reusing good open source}
	\entry
	  {2017 -- 2018\\\footnotesize{part time}}
	  {Game developer}
	  {Project Nyama}
	  {Working as a Game developer with some friends from University.
	  My role was to program the enemy's tracking and attack algorithms.\\ \texttt{Unity}\slashsep\texttt{C\#}}
	\entry
	  {1998 -- 2013}
	  {A-Levels}
	  {The Lady Elizabeth School}
	  {I went to this English school when I was 3 years old and finished studying there at 17, afterwards, I went to University.}
\end{entrylist}


\end{document}
