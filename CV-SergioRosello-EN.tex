%%%%%%%%%%%%%%%%%%%%%%%%%%%%%%%%%%%%%%%%%
% Developer CV
% LaTeX Template
% Version 1.0 (28/1/19)
%
% This template originates from:
% http://www.LaTeXTemplates.com
%
% Authors:
% Jan Vorisek (jan@vorisek.me)
% Based on a template by Jan Küster (info@jankuester.com)
% Modified for LaTeX Templates by Vel (vel@LaTeXTemplates.com)
%
% License:
% The MIT License (see included LICENSE file)
%
%%%%%%%%%%%%%%%%%%%%%%%%%%%%%%%%%%%%%%%%%

%----------------------------------------------------------------------------------------
%	PACKAGES AND OTHER DOCUMENT CONFIGURATIONS
%----------------------------------------------------------------------------------------

\documentclass[9pt]{developercv} % Default font size, values from 8-12pt are recommended

%----------------------------------------------------------------------------------------

\begin{document}

%----------------------------------------------------------------------------------------
%	TITLE AND CONTACT INFORMATION
%----------------------------------------------------------------------------------------

\begin{minipage}[t]{0.45\textwidth} % 45% of the page width for name
	\vspace{-\baselineskip} % Required for vertically aligning minipages
	
	% If your name is very short, use just one of the lines below
	% If your name is very long, reduce the font size or make the minipage wider and reduce the others proportionately
	\colorbox{black}{{\HUGE\textcolor{white}{\textbf{\MakeUppercase{Sergio}}}}} % First name
	
	\colorbox{black}{{\HUGE\textcolor{white}{\textbf{\MakeUppercase{Roselló}}}}} % Last name
	
	\vspace{6pt}
	
	{\huge Software Engineer} % Career or current job title
\end{minipage}
\begin{minipage}[t]{0.275\textwidth} % 27.5% of the page width for the first row of icons
	\vspace{-\baselineskip} % Required for vertically aligning minipages
	
	% The first parameter is the FontAwesome icon name, the second is the box size and the third is the text
	% Other icons can be found by referring to fontawesome.pdf (supplied with the template) and using the word after \fa in the command for the icon you want
	\icon{MapMarker}{12}{Madrid}\\
	\icon{Phone}{12}{+34 685106958}\\
	\icon{At}{12}{\href{mailto:sergio-rosello@hotmail.com}{sergio-rosello@h..l.com}}\\	
\end{minipage}
\begin{minipage}[t]{0.275\textwidth} % 27.5% of the page width for the second row of icons
	\vspace{-\baselineskip} % Required for vertically aligning minipages
	
	% The first parameter is the FontAwesome icon name, the second is the box size and the third is the text
	% Other icons can be found by referring to fontawesome.pdf (supplied with the template) and using the word after \fa in the command for the icon you want
	\icon{Globe}{12}{\href{http://www.sergiorosello.com}{sergiorosello.com}}\\
	\icon{Github}{12}{\href{https://github.com/sergiorosello}{github.com/sergiorosello}}\\
	\icon{Instagram}{12}{\href{https://www.instagram.com/sergiorosello/}{sergiorosello}}\\
\end{minipage}

\vspace{0.5cm}

%----------------------------------------------------------------------------------------
%	INTRODUCTION, SKILLS AND TECHNOLOGIES
%----------------------------------------------------------------------------------------

\cvsect{Who Am I?}

\begin{minipage}[t]{0.4\textwidth} % 40% of the page width for the introduction text
	\vspace{-\baselineskip} % Required for vertically aligning minipages
	
	I am a very curious guy, passionate about learning daily.
	I'm interested in life and the lessons I can learn from it.
	Attitudes like teamwork, perseverance and passion define me as a person.
	Aptitudes are developed from that foundation onward.
	I value spending time with myself and get to know and shape who I want to become.\\
\end{minipage}
\hfill % Whitespace between
\begin{minipage}[t]{0.5\textwidth} % 50% of the page for the skills bar chart
	\vspace{-\baselineskip} % Required for vertically aligning minipages
	\begin{barchart}{5.5}
		\baritem{Bash}{60}
		\baritem{C/C++}{40}
		\baritem{Git}{80}
		\baritem{AWS}{50}
		\baritem{GNU/Linux}{60}
		\baritem{Java}{70}
		\baritem{Vim}{70}
		\baritem{Ruby}{60}
	\end{barchart}
\end{minipage}

%----------------------------------------------------------------------------------------
%	EXPERIENCE
%----------------------------------------------------------------------------------------

\cvsect{Experience}

\begin{entrylist}
	\entry
		{2018 -- Current\\\footnotesize{Full time}}
		{QA Engineer}
		{Wave Location Technologies}
		{My role is to ensure the quality of every app under the company's name.
		To accomplish this, I have implemented a end-to-end test automation framework. I also do manual testing.\\ \texttt{Ruby}\slashsep\texttt{Bash}\slashsep\texttt{Jenkins}\slashsep\texttt{Linux}\slashsep\texttt{Gherkin}\slashsep\texttt{Appium}}
	\entry
		{2017 -- 2018\\\footnotesize{part time}}
		{Game developer}
		{Project Nyama}
		{Working as a Game developer with some friends from University.
		My role was to program the enemy's tracking and attack algorithms.\\ \texttt{Unity}\slashsep\texttt{C\#}}
\end{entrylist}

%----------------------------------------------------------------------------------------
%	EDUCATION
%----------------------------------------------------------------------------------------

\cvsect{Education}

\begin{entrylist}
	\entry
		{2014 -- 2018}
		{Bachelor's Degree}
		{U-TAD}
		{The Degree is very practically focused. This means I have had a hands on approach to everything I have learned here.
		My Bachelor's Degree Final Project was focused on an analysis of double authentication methods on GNU/Linux systems with a special focus on PAM}
	\entry
		{1998 -- 2013}
		{A-Levels}
		{The Lady Elizabeth School}
		{I went to this school when I was 3 years old and finished studying there at 17, when I went to University. This is a English School, meaning I have grown up with English around me constantly.}
\end{entrylist}

\cvsect{Publicaciones}

%----------------------------------------------------------------------------------------
%	Publications
%----------------------------------------------------------------------------------------
\begin{entrylist}
	\entry
		{25-01-2019}
		{Setup an automated testing server with Jenkins and AWS Device farm}
		{Faun - Medium}
		{Step by step configuration of a Jenkins, AWS - Device Fram CI server}
	\entry
		{01-12-2018}
		{Leverage the power of open source and don’t reinvent the wheel}
		{Hacker Noon - Medium}
    {How not to waste time reusing good open source}
\end{entrylist}

%----------------------------------------------------------------------------------------
%	MINDSET
%----------------------------------------------------------------------------------------
\cvsect{Personality}

\begin{entrylist}
	\entry
		{Currently}
		{ISFP}
		{Myers and Briggs'}
		{I focus on the job at hand, always prioritising people relations first.
		I want everybody to feel secure and accepted.
		I tend to pause for a better answer.
		Self-improvement is always on my To-Do list.
		Less is more.}
\end{entrylist}

%----------------------------------------------------------------------------------------
%	ADDITIONAL INFORMATION
%----------------------------------------------------------------------------------------

\begin{minipage}[t]{0.3\textwidth}
	\vspace{-\baselineskip} % Required for vertically aligning minipages

	\cvsect{Languages}
	
	\textbf{Spanish} - native\\
	\textbf{Catalan} - native\\
	\textbf{English} - C1 - Cambridge
\end{minipage}
\hfill
\begin{minipage}[t]{0.3\textwidth}
	\vspace{-\baselineskip} % Required for vertically aligning minipages
	
	\cvsect{Hobbies}
	
	
	I play around with GNU/Linux (Arch + i3) and configure my system to be more efficient.
	Whenever I need time for myself, I will go for a walk. This helps me organise my ideas.
	I also enjoy photography, I think of it as my creative outburst.
	When I need a bit of an adrenaline rush, I will go for a ride with my bicicle.
\end{minipage}
\hfill
\begin{minipage}[t]{0.3\textwidth}
	\vspace{-\baselineskip} % Required for vertically aligning minipages
	
	\cvsect{OPEN SOURCE}
	
	I believe Software should be Open Source as it serves as a learning platform for aspiring developers as much as a community effort to break current boundaries and limitations. In our day and age, no commercial product would be feasible without relying on quality Open Source projects.
\end{minipage}

%----------------------------------------------------------------------------------------

\end{document}
