%%%%%%%%%%%%%%%%%%%%%%%%%%%%%%%%%%%%%%%%%
% Developer CV
% LaTeX Template
% Version 1.0 (28/1/19)
%
% This template originates from:
% http://www.LaTeXTemplates.com
%
% Authors:
% Jan Vorisek (jan@vorisek.me)
% Based on a template by Jan Küster (info@jankuester.com)
% Modified for LaTeX Templates by Vel (vel@LaTeXTemplates.com)
%
% License:
% The MIT License (see included LICENSE file)
%
%%%%%%%%%%%%%%%%%%%%%%%%%%%%%%%%%%%%%%%%%

%----------------------------------------------------------------------------------------
%	PACKAGES AND OTHER DOCUMENT CONFIGURATIONS
%----------------------------------------------------------------------------------------

\documentclass[9pt]{developercv} % Default font size, values from 8-12pt are recommended

%----------------------------------------------------------------------------------------

\begin{document}

%----------------------------------------------------------------------------------------
%	TITLE AND CONTACT INFORMATION
%----------------------------------------------------------------------------------------

\begin{minipage}[t]{0.45\textwidth} % 45% of the page width for name
	\vspace{-\baselineskip} % Required for vertically aligning minipages
	
	% If your name is very short, use just one of the lines below
	% If your name is very long, reduce the font size or make the minipage wider and reduce the others proportionately
	\colorbox{black}{{\HUGE\textcolor{white}{\textbf{\MakeUppercase{Sergio}}}}} % First name
	
	\colorbox{black}{{\HUGE\textcolor{white}{\textbf{\MakeUppercase{Roselló}}}}} % Last name
	
	\vspace{6pt}
	
	{\huge Ingeniero de Software} % Career or current job title
\end{minipage}
\begin{minipage}[t]{0.275\textwidth} % 27.5% of the page width for the first row of icons
	\vspace{-\baselineskip} % Required for vertically aligning minipages
	
	% The first parameter is the FontAwesome icon name, the second is the box size and the third is the text
	% Other icons can be found by referring to fontawesome.pdf (supplied with the template) and using the word after \fa in the command for the icon you want
	\icon{MapMarker}{12}{Madrid}\\
	\icon{Phone}{12}{+34 685106958}\\
	\icon{At}{12}{\href{mailto:sergio-rosello@hotmail.com}{sergio-rosello}}\\	
\end{minipage}
\begin{minipage}[t]{0.275\textwidth} % 27.5% of the page width for the second row of icons
	\vspace{-\baselineskip} % Required for vertically aligning minipages
	
	% The first parameter is the FontAwesome icon name, the second is the box size and the third is the text
	% Other icons can be found by referring to fontawesome.pdf (supplied with the template) and using the word after \fa in the command for the icon you want
	\icon{Globe}{12}{\href{http://www.sergiorosello.com}{sergiorosello.com}}\\
	\icon{Github}{12}{\href{https://github.com/sergiorosello}{github.com/sergiorosello}}\\
	\icon{Instagram}{12}{\href{https://www.instagram.com/sergiorosello/}{sergiorosello}}\\
\end{minipage}

\vspace{0.5cm}

%----------------------------------------------------------------------------------------
%	INTRODUCTION, SKILLS AND TECHNOLOGIES
%----------------------------------------------------------------------------------------

\cvsect{Quien soy?}

\begin{minipage}[t]{0.4\textwidth} % 40% of the page width for the introduction text
	\vspace{-\baselineskip} % Required for vertically aligning minipages
	Soy una persona muy curiosa, me apasiona aprender nuevas cosas diariamente.
	Me interesa la vida y lo que podemos averiguar viviéndola.
	Actitudes como el trabajo en equipo, la perseverancia o pasión me definen como persona.
	Las aptitudes se desarrollan a partir de unos buenos cimientos.
	Valoro mi tiempo y lo invierto, entre otras cosas, en descubrir en quien me quiero convertir.\\
	
\end{minipage}
\hfill % Whitespace between
\begin{minipage}[t]{0.5\textwidth} % 50% of the page for the skills bar chart
	\vspace{-\baselineskip} % Required for vertically aligning minipages
	\begin{barchart}{5.5}
		\baritem{Bash}{60}
		\baritem{C/C++}{40}
		\baritem{Git}{80}
		\baritem{AWS}{50}
		\baritem{GNU/Linux}{60}
		\baritem{Java}{70}
		\baritem{Vim}{70}
		\baritem{Ruby}{60}
	\end{barchart}
\end{minipage}

%----------------------------------------------------------------------------------------
%	EXPERIENCE
%----------------------------------------------------------------------------------------

\cvsect{Experiencia}

\begin{entrylist}
	\entry
		{2018 -- Actualidad\\\footnotesize{A tiempo completo}}
		{Ingeniero QA}
		{Wave Location Technologies}
		{Mi tarea es asegurar la calidad de todas las aplicaciones que saca la empresa.
		Para conseguirlo, he implementado un framework de pruebas end-to-end automático.
		También hago pruebas manuales.\\ \texttt{Ruby}\slashsep\texttt{Bash}\slashsep\texttt{Jenkins}\slashsep\texttt{Linux}\slashsep\texttt{Gherkin}\slashsep\texttt{Appium}}
	\entry
		{2017 -- 2018\\\footnotesize{A tiempo parcial}}
		{Desarrollador de vídeo juegos}
		{Project Nyama}
		{Trabajaba como desarrollador en un vídeo juego con unos amigos de la carrera.
		Me encargaba del comportamiento de persecución y ataque del enemigo.\\ \texttt{Unity}\slashsep\texttt{C\#}}
\end{entrylist}

%----------------------------------------------------------------------------------------
%	EDUCATION
%----------------------------------------------------------------------------------------

\cvsect{Educación}

\begin{entrylist}
	\entry
		{2014 -- 2018}
		{Grado en Ingeniaría de Software}
		{U-TAD}
		{El grado aquí sigue unas directrices muy practicas por tanto he hecho todo lo que he aprendido.
		En el TFG, me centré en el análisis de factores de doble autenticación en sistemas GNU/Linux. Ahondando en la investigacion de PAM}
	\entry
		{1998 -- 2013}
		{A-Levels}
		{The Lady Elizabeth School}
    {He ido a esta escuela inglesa desde que tenia tres años hasta los 17. El ingles siempre ha formado parte de mi día a día.}
\end{entrylist}

\cvsect{Publicaciones}

\begin{entrylist}
	\entry
		{25-01-2019}
		{Setup an automated testing server with Jenkins and AWS Device farm}
		{Faun - Medium}
		{Configuro un servidor de pruebas end-to-end con Jenkins y AWS Device Farm}
	\entry
		{01-12-2018}
		{Leverage the power of open source and don’t reinvent the wheel}
		{Hacker Noon - Medium}
    {Como usar proyectos de código libre para conseguir lo que necesitas, sin reinventar la rueda}
\end{entrylist}

%----------------------------------------------------------------------------------------
%	MINDSET
%----------------------------------------------------------------------------------------
\cvsect{Personalidad}

\begin{entrylist}
	\entry
		{Actualmente}
		{ISFP}
		{Myers and Briggs'}
		{Me centro en el proyecto, pero siempre priorizando las relaciones humanas antes.
		Quiero que todo el mundo se sienta seguro y aceptado.
		Pienso antes de responder.
		La mejora personal siempre está en mi lista de cosas a hacer.
		Menos es mas.}
\end{entrylist}

%----------------------------------------------------------------------------------------
%	ADDITIONAL INFORMATION
%----------------------------------------------------------------------------------------

\begin{minipage}[t]{0.3\textwidth}
	\vspace{-\baselineskip} % Required for vertically aligning minipages

	\cvsect{Idiomas}
	
	\textbf{Español} - Lengua materna\\
	\textbf{Catalán} - Lengua materna\\
	\textbf{Inglés} -  C1 - Cambridge
\end{minipage}
\hfill
\begin{minipage}[t]{0.3\textwidth}
	\vspace{-\baselineskip} % Required for vertically aligning minipages
	
	\cvsect{Hobbies}
	
  Optimizo mi sistema GNU-Linux (Arch + i3) configurándolo para que sea más eficiente.
  Andar me ayuda a organizar mis ideas cuando necesito dedicarme un tiempo.
  Disfruto de la fotografía, pienso que es mi expresión creativa.
  Cuando necesito soltar adrenalina, salgo a dar una vuelta en mi bicicleta.
\end{minipage}
\hfill
\begin{minipage}[t]{0.3\textwidth}
	\vspace{-\baselineskip} % Required for vertically aligning minipages
	
	\cvsect{CÓDIGO ABIERTO}

	Creo que el software debería ser código abierto ya que ayuda a nuevos desarrolladores a aprender y a la comunidad a expandir el conocimiento colectivo. 
\end{minipage}

%----------------------------------------------------------------------------------------

\end{document}
